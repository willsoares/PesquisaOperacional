\documentclass[a4paper,10pt]{article}
\usepackage[utf8]{inputenc}
\usepackage[english]{babel}

%opening
\title{Report 2 \textendash\ Operations Research}
\author{Willian de Souza Soares \textendash\ 2014.1.08.034}
\date{March 9, 2020}

\begin{document}
\maketitle

\pagenumbering{gobble}
\begin{abstract}
A simple presentation of the solutions and thoughts about \emph{Container's} problem, using \emph{CPLEX} and \emph{Java}.
\end{abstract}

\section{Finding the optimal values}
The majority of the instances were easy and fast to find it's optimal value. On the instances \emph{00} and \emph{01}, the optimal value were found both in Integer and Real models. In contrast, only the Real optimal value was found on the \emph{02} instance. 

\subsection{The problem with \emph{Instance 02}}

One of the problems with the instance 02 it's the size. Whensoever the instances \emph{00} and \emph{01} has, respectively, 10 and 50 items, the instance \emph{02} has 500 items, with mixed profit, weight and volume. 

Although the size contribute to increase the complexity of the items, the major factor it's the \emph{b} value, which indicates the maximum number of times that same item can be put in a container. This increase the complexity by many magnitudes, slowing down the algorithm, specially on the Integer solution, which the \emph{CPLEX solver} couldn't find the optimal solution in a reasonable time.

\subsection{The solution}

The simplest solution was taken: A time limit to find the optimal solution was set. An arbitrary value of 600 seconds (10 minutes) was set. Since the \emph{gap} of the optimal solution was under 1\%, it was an acceptable value, given the nature of the problem.

\begin{table}[h]
\caption{Results for a Integer model}
\centering
\label{tab:my-table}
\begin{tabular}{ccccc}
\hline
Instance & Profit             & Gap    & Optimal & Time Limit \\ \hline
00       & 42.92491516324943  & 0\%    & Yes     & N/A       \\
01       & 91.3135124472273   & 0\%    & Yes     & N/A       \\
02       & 508.26966170662337 & 0.61\% & No      & 600s      \\ \hline
\end{tabular}
\end{table}

\newpage


\begin{table}[h]
\caption{Results for a Real model}
\centering
\label{tab:my-table1}
\begin{tabular}{ccccc}
\hline
Instance & Profit             & Gap & Optimal & Time Limit \\ \hline
00       & 49.571868896306285 & 0\% & Yes     & N/A       \\
01       & 95.89502739170803  & 0\% & Yes     & N/A       \\
02       & 511.45921634552246 & 0\% & Yes     & N/A       \\ \hline
\end{tabular}
\end{table}

\section{Conclusions}
The problem that once was rather hard to find a solution, now it's simple with the mathematical modeling of \emph{CPLEX}. Given the circumstances, the difference between an sub-optimal value that is found in minutes is better than a optimal one.

\end{document}
