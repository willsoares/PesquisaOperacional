\documentclass[10pt,a4paper]{article}
\usepackage[utf8]{inputenc}
\usepackage[english]{babel}
\usepackage{amsmath}
\usepackage{amsfonts}
\usepackage{amssymb}
\author{Willian de Souza Soares \textendash\ 2014.1.08.034}
\title{Report 3 \textendash\ Operations Research}
\begin{document}
\maketitle
\pagenumbering{gobble}
\begin{abstract}
A new implementation on the container's problem, modeling based in a three-dimensional matrix and boolean domain.
\end{abstract}
\section{The difference between approaches}
The major difference between the approach described on the Report 2 and this is the usage of an three-dimensional matrix to represent the solution, modeling the objective function around having a boolean solution.

\section{Results}

\begin{table}[h]
\caption{Results for a Integer solution, \emph{Report 2} implementation}
\centering
\label{tab:my-table1}
\begin{tabular}{cccccc}
\hline
Instance & Profit             & Gap    & Optimal & Time Elapsed & Number of Ticks 	\\ \hline
00       & 42.92491516324943  & 0\%    & Yes     & 0.03s       	& 0.50 ticks	\\
01       & 91.3135124472273   & 0\%    & Yes     & 0.37s      	& 43.36 ticks	\\
02       & 504.24308349895546 & 1.41\% & No      & 60s*			& 21874.30 ticks\\ \hline
\end{tabular}
\end{table}

\begin{table}[h]
\caption{Results for a Integer solution, Three-Dimensional boolean solution}
\centering
\label{tab:my-table1}
\begin{tabular}{cccccc}
\hline
Instance & Profit             & Gap    & Optimal & Time Elapsed & Number of Ticks 	\\ \hline
00       & 42.92491516324943  & 0\%    & Yes     & 0.03s       	& 0.57 ticks	\\
01       & 91.3135124472273   & 0\%    & Yes     & 0.32s      	& 43.78 ticks	\\
02       & 504.24308349895546 & 2.72\% & No      & 60s*			& 24689.29 ticks\\ \hline
\end{tabular}
\end{table}
*The instance 02 reached the time limit of 60 seconds.
\end{document}