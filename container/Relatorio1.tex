%%%%%%%%%%%%  Generated using docx2latex.com  %%%%%%%%%%%%%%

%%%%%%%%%%%%  v2.0.0-beta  %%%%%%%%%%%%%%

\documentclass[12pt]{article}
\usepackage{amsmath}
\usepackage{latexsym}
\usepackage{amsfonts}
\usepackage[normalem]{ulem}
\usepackage{soul}
\usepackage{array}
\usepackage{amssymb}
\usepackage{extarrows}
\usepackage{graphicx}
\usepackage[backend=biber,
style=numeric,
sorting=none,
isbn=false,
doi=false,
url=false,
]{biblatex}\addbibresource{bibliography.bib}

\usepackage{subfig}
\usepackage{wrapfig}
\usepackage{wasysym}
\usepackage{enumitem}
\usepackage{adjustbox}
\usepackage{ragged2e}
\usepackage[svgnames,table]{xcolor}
\usepackage{tikz}
\usepackage{longtable}
\usepackage{changepage}
\usepackage{setspace}
\usepackage{hhline}
\usepackage{multicol}
\usepackage{tabto}
\usepackage{float}
\usepackage{multirow}
\usepackage{makecell}
\usepackage{fancyhdr}
\usepackage[toc,page]{appendix}
\usepackage[hidelinks]{hyperref}
\usetikzlibrary{shapes.symbols,shapes.geometric,shadows,arrows.meta}
\tikzset{>={Latex[width=1.5mm,length=2mm]}}
\usepackage{flowchart}\usepackage[paperheight=11.69in,paperwidth=8.27in,left=0.79in,right=0.79in,top=0.79in,bottom=0.79in,headheight=1in]{geometry}
\usepackage[utf8]{inputenc}
\usepackage[T1]{fontenc}

\usepackage{listings}
\usepackage{xcolor}

\lstdefinestyle{sharpc}{language=[Sharp]C,tabsize=2,frame=lr, rulecolor=\color{blue!80!black}}

\TabPositions{0.49in,0.98in,1.47in,1.96in,2.45in,2.94in,3.43in,3.92in,4.41in,4.9in,5.39in,5.88in,6.37in,}

\urlstyle{same}

\renewcommand{\_}{\kern-1.5pt\textunderscore\kern-1.5pt}

 %%%%%%%%%%%%  Set Depths for Sections  %%%%%%%%%%%%%%

% 1) Section
% 1.1) SubSection
% 1.1.1) SubSubSection
% 1.1.1.1) Paragraph
% 1.1.1.1.1) Subparagraph


\setcounter{tocdepth}{5}
\setcounter{secnumdepth}{5}


 %%%%%%%%%%%%  Set Depths for Nested Lists created by \begin{enumerate}  %%%%%%%%%%%%%%


\setlistdepth{9}
\renewlist{enumerate}{enumerate}{9}
		\setlist[enumerate,1]{label=\arabic*)}
		\setlist[enumerate,2]{label=\alph*)}
		\setlist[enumerate,3]{label=(\roman*)}
		\setlist[enumerate,4]{label=(\arabic*)}
		\setlist[enumerate,5]{label=(\Alph*)}
		\setlist[enumerate,6]{label=(\Roman*)}
		\setlist[enumerate,7]{label=\arabic*}
		\setlist[enumerate,8]{label=\alph*}
		\setlist[enumerate,9]{label=\roman*}

\renewlist{itemize}{itemize}{9}
		\setlist[itemize]{label=$\cdot$}
		\setlist[itemize,1]{label=\textbullet}
		\setlist[itemize,2]{label=$\circ$}
		\setlist[itemize,3]{label=$\ast$}
		\setlist[itemize,4]{label=$\dagger$}
		\setlist[itemize,5]{label=$\triangleright$}
		\setlist[itemize,6]{label=$\bigstar$}
		\setlist[itemize,7]{label=$\blacklozenge$}
		\setlist[itemize,8]{label=$\prime$}

\setlength{\topsep}{0pt}\setlength{\parindent}{0pt}
\renewcommand{\arraystretch}{1.3}

\title{Pesquisa Operacional – Relatório 1}
\author{Willian de Souza Soares}
\date{}


%%%%%%%%%%%%%%%%%%%% Document code starts here %%%%%%%%%%%%%%%%%%%%



\begin{document}

\maketitle
\setlength{\parskip}{6.0pt}
\par


\par


\vspace{\baselineskip}
{\fontsize{13pt}{15.6pt}\selectfont \tab Foi utilizada um implementação básica de ranqueamento dos itens a serem carregados.\par}\par

{\fontsize{13pt}{15.6pt}\selectfont \tab Comecei tentando utilizar Lucro$  \div $Peso do item para ranquear os items, o que obviamente não serviria: alguns itens tem um Lucro alto e peso baixo, porém tem o Volume altíssimo, o que os colocariam em uma posição pior no ranking. Alterei a fórmula para Lucro$  \div $Peso$  \div $Volume, que\  fazia bem mais sentido. Assim, após ordenar esta lista de itens, temos um roteiro básico de quais itens compensam mais ser carregados em um container.\par}\par

{\fontsize{13pt}{15.6pt}\selectfont \tab Apesar de acreditar que o ranqueamento está aceitável por hora, algumas coisas ficaram para trás, como aproveitar completamente o espaço de um container. Por diversas vezes, o algoritmo não completa os container por falta de uma lógica que os permita continuar procurando itens menos valiosos que talvez completem as lacunas de um container.\par}\par

{\fontsize{13pt}{15.6pt}\selectfont \tab Certamente, dada uma outra oportunidade, mudaria algumas coisas no projeto: o ranqueamento precisa de algum toque, principalmente quando se trata de uma instancia que pode-se fracionar os itens; o sistema de preechimento dos containers é muito falho e faltam diversas lógicas para aproveitar os espaços vazios. O mal aproveitamento dos containeres chega a levar o lucro da instância 2 em apenas 317.85, o que em comparação com outras soluções chega a ser uma falta de 100-200 de lucro.\par}\par


\vspace{\baselineskip}
{\fontsize{13pt}{15.6pt}\selectfont Em seguida, trexos relevantes da classe Logistica, que contém o algoritmo de organização de items e \textit{assignment} aos containers.\par}\par


\lstset{style=sharpc}
\begin{lstlisting}[basicstyle=\tiny]
public List<Container> OtimizaLucro()
{
	List<(int qtdItemFrete, Item item)> itemsOrganizados = OrdenaItems().Select(x => (0, x.Item2)).ToList();
	for (int i = 0; i < itemsOrganizados.Count; i++)
	{
		for (int c = 0; c < Frete.Containers.Count; c++)
		{
			if ((Frete.CargaMax >= Frete.Containers[c].Carga + itemsOrganizados[i].item.Peso) &&
			(Frete.VolumeMax >= Frete.Containers[c].Volume + itemsOrganizados[i].item.Volume))
			{
				if (itemsOrganizados[i].qtdItemFrete < Frete.QtdMaxItem)
				{
					Frete.Containers[c].Items.Add(itemsOrganizados[i].item);
					itemsOrganizados[i] = (itemsOrganizados[i].qtdItemFrete + 1, itemsOrganizados[i].item);
				}
			}
		}
	}
	return Frete.Containers;
}
public List<(double, Item)> OrdenaItems()
{
	var items = new List<(double beneficio, Item item)>();
	foreach (var item in Frete.ItemsDisponiveis)
	{
		items.Add((item.Lucro / item.Peso / item.Volume, item));
	}
	items.Sort((x, y) => y.beneficio.CompareTo(x.beneficio));
	return items;
}
\end{lstlisting}
\vspace{\baselineskip}

\printbibliography
\end{document}